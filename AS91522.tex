\documentclass[11pt, a4paper]{article}
\usepackage[margin=1cm]{geometry}
\usepackage[utf8]{inputenc}
\usepackage{hyperref, amsfonts,amssymb,amsmath,tikz,pgfplots}
\usetikzlibrary{positioning}

\title{\small AS91522 - Physics 3.2\\ \huge Circular Motion of a Stunt Glider}
\author{Nathan Tasker}
\date{\today}

\begin{document}
	\maketitle
	\tableofcontents
	\newpage
	\section{Vertical Circle: Loop the loop including Dip and Arch}
	\subsection{Achieved}
	During motion, 3 forces are acting on the stunt glider:
	\begin{enumerate}
		\item Gravity ($\vec{F_g}$) (i.e. Weight) always vertically downwards (i.e. toward center of Earth).
		\item Lift ($\vec{F_L}$) perpendicular to direction of velocity, toward the center of the circular path.
		\item Air Resistance ($\vec{F_R}$) (i.e. Friction, Drag) opposite to direction of velocity.
	\end{enumerate}
	\begin{center}
		\begin{tikzpicture}
			\pgfmathsetmacro{\MaxVelocityLength}{2}
			\pgfmathsetmacro{\MinVelocityLength}{1}
			\pgfmathsetmacro{\GravityLength}{1}
			\pgfmathsetmacro{\PathRadius}{5}
			\pgfmathsetmacro{\GliderPointRadius}{0.1}
			\coordinate(TopPos) at (0,\PathRadius);
			\coordinate(BottomPos) at (0,-\PathRadius);
			
			\draw[dashed, gray] (0,0) circle (\PathRadius);
			
			\filldraw[black] (TopPos) circle (\GliderPointRadius);
			\draw[->] (TopPos) -- ++(-\MinVelocityLength,0) node[left] {$\vec{v}$};
			\draw[->] (TopPos) -- ++(0,-\GravityLength) node[below] {$\vec{F_g}$};
			
			\filldraw[black] (BottomPos) circle (\GliderPointRadius);
			\draw[->] (BottomPos) -- ++(\MaxVelocityLength,0) node[right] {$\vec{v}$};
			\draw[->] (BottomPos) -- ++(0,-\GravityLength) node[below] {$\vec{F_g}$};
			\draw[->] (BottomPos) -- ++(0,2) node[above] {$\vec{F_L}$};
		\end{tikzpicture}
	\end{center}
	For simplicity, air resistance will be ignored as its effects are negligible?\\
	The force of gravity is always constant in both magnitude and direction regardless of the glider's velocity or position in the loop (top, bottom, and anywhere else).\\
	Conversely, the force of lift varies in both direction and magnitude depending on these factors.\\
	Its magnitude is greatest at the bottom of the circular path, because to maintain circular motion the centripetal force ($\vec{F_c}=\vec{F_L}+\vec{F_g}$) towards the center of the circular path must have a magnitude value greater than zero ($|\vec{F_c}|>0N$) (and also great enough to provide centripetal acceleration for circular path radius), meaning the lift force upwards must be greater than the gravity force downwards.\\
	It's magnitude is least at the top of the circular path, because the force of gravity already provides 
	
	\subsection{Merit}
	\subsection{Excellence}
	\section{Banked Corner}
	\subsection{Achieved}
	\subsection{Merit}
	\subsection{Excellence}
	\section{Additional Info}
	\subsection{Comprehensive Version History}
	Access to all prior versions of this document during process of creation is publicly available at:\\
	\url{https://github.com/NathanTaskerPersonal/AS91522}
	\subsection{Graphical Analysis Files}
	Access to all graphical analysis files are publically available at:\\
	\href{https://middletonschoolnz-my.sharepoint.com/:f:/g/personal/taskern_middleton_school_nz/EhEmw21C2L9Fn9BYUy2ccwMBn6xCUF93vtfvtT_5_rkxbA?e=Tp02lP}{middletonschoolnz-my.sharepoint.com/...}
	\subsection{Bibliography}
\end{document}
