\documentclass[11pt, a4paper]{article}
\usepackage[left=1cm, right=1cm, top=1cm, bottom=2cm]{geometry}
\parindent 0px % indent new lines by 0 pixels
\usepackage[utf8]{inputenc}
\usepackage{hyperref, amsfonts,amssymb,amsmath,tikz,pgfplots,float,graphicx}
\usetikzlibrary{positioning}

\def\ParagraphSpacing{30pt}

\title{\small AS91522 - Physics 3.2\\ \huge Circular Motion of a Stunt Glider}
\author{Nathan Tasker}
\date{\today}

\begin{document}
	\maketitle
	\tableofcontents
	\newpage
	\section{Vertical Circle: Loop the loop including Dip and Arch}
	\subsection{Achieved}
	The force of gravity is always constant in both magnitude and direction regardless of the glider's velocity or position in the loop (top, bottom, and anywhere else). This is because $|\vec{F_g}|=mg$ where mass ($m$) and the acceleration of gravity ($g$) are constants.\\[\ParagraphSpacing]
	Conversely, the force of lift varies in both direction and magnitude as the glider performs the loop the loop.\\
	Lift's magnitude ($|\vec{F_L}|$) is greatest at the bottom, $\because |\vec{F_L}|=\frac{1}{2}p|\vec{v}|^2AC_L$, $\therefore |\vec{F_L}|\propto v^2$. Because velocity is greatest at bottom, lift force is as well. To maintain circular motion the centripetal force ($|\vec{F_c}|=|\vec{F_L}|-|\vec{F_g}|$) towards the center of the circular path must have a magnitude value great enough to provide necessary centripetal acceleration for the circular path radius, meaning the lift force upwards must at least be greater than the gravity force downwards ($|\vec{F_L}|>|\vec{F_g}|$ in order for $|\vec{F_c}|>0$).\\
	Lift's magnitude ($|\vec{F_L}|$) is least at the top, because the direction of gravity force is toward the center of the circular path. This means ($|\vec{F_c}|=|\vec{F_L}|+|\vec{F_g}|$).\\
	\begin{figure}[H]
		\centering
		\begin{tikzpicture}
			\pgfmathsetmacro{\TextGap}{0.2};
			\pgfmathsetmacro{\HalfWidth}{8};
			\pgfmathsetmacro{\TopLineY}{3.05};
			\pgfmathsetmacro{\BottomLineY}{-2.65};
			\coordinate(BottomPos1) at (-5.65,\TopLineY);
			\coordinate(TopPos) at (-0.1,-2.58);
			\coordinate(BottomPos2) at (5.6,2.9);
			\node[inner sep=0] at (0,0) {\includegraphics[width=0.8\textwidth]{Images/Loop the loop graph.png}};
			\draw[black, dashed] (-\HalfWidth,\TopLineY) -- (\HalfWidth,\TopLineY);
			\draw[black, dashed] (-\HalfWidth,\BottomLineY) -- (\HalfWidth,\BottomLineY);
			\filldraw[red!50!black] (BottomPos1) circle (.1) node[above=\TextGap, align=center] {Bottom of the vertical circle\\(15.82s, 5.89N)};
			\filldraw[green!50!black] (TopPos) circle (.1) node[above=\TextGap, align=center] {Top of the vertical circle\\(16.30s, 1.13N)};
			\filldraw[blue!50!black] (BottomPos2) circle (.1) node[above=\TextGap, align=center] {Bottom of the vertical circle\\(16.80s, 5.81N)};
		\end{tikzpicture}
		\caption{Graph of a complete vertical loop}
	\end{figure}
	At the bottom of the vertical circle, the tension force (emulating the lift force, $\vec{F_L}$) is greatest, which occurs at 15.82 seconds with 5.89N.\\
	Once the force meter (emulating motion of glider) reaches the top of the vertical circle, the tension force is at its least, which - when smoothing the curve of data and reducing random variation/noise - occurs at\\
	Check if sine model or quadratic helps - then return to this section.\\
	During the loop the loop, as the glider ascends, its kinetic energy ($E_K$) is converted into gravitational potential energy ($E_p$). As the glider descends, its gravitational potential energy ($E_p$) is converted back into kinetic energy ($E_K$).\\
	
	\subsection{Merit}
	\begin{align}
		E_K&=\frac{1}{2}m|\vec{v}|^2\\
		\therefore |\vec{v}|^2&\propto E_K\\
		|\vec{v}|&\propto \sqrt{E_K}
	\end{align}
	During the loop the loop, as the glider ascends, its kinetic energy ($E_K$) is converted into gravitational potential energy ($E_p$). This decreases its velocity as it is proportional to kinetic energy, which decreased when converted to gravitational potential energy. \\
	As the glider descends, its gravitational potential energy ($E_p$) is converted back into kinetic energy ($E_K$).This increases its velocity as it is proportional to kinetic energy, which increased when the gravitational potential energy was converted back into it.\\[\ParagraphSpacing]
	A glider's ability to successfully perform a vertical loop relies on its ability to provide an adequate lift force, increasing both net and centripital force.
	\begin{align}
		|\vec{F_L}|&=\frac{1}{2}p|\vec{v}|^2AC_L\\
		\therefore |\vec{F_L}|&\propto |\vec{v}|^2
	\end{align}
	Equation (5) shows that velocity is a major factor in the magnitude of the lift force. Without adequate velocity, inadequate lift force is provided for the vertical loop to be successful. As kinetic energy is converted to gravitational potential energy during ascent, if the velocity drops below what is required for the circular path radius its motion will instead imitate that of a projectile.\\
	Other factors proportional to velocity include surface area contacting air ($A$), and the coefficient of lift ($C_L$), determined by shape and angle of attack).\\
	Fluid density ($p$) is constant.\\[\ParagraphSpacing]
	During motion, 3 forces are acting on the stunt glider:
	\begin{enumerate}
		\item Gravity ($\vec{F_g}$) (i.e. Weight) always vertically downwards (i.e. toward center of Earth).
		\item Lift ($\vec{F_L}$) perpendicular to direction of velocity, toward the center of the circular path.
		\item Air Resistance ($\vec{F_R}$) (i.e. Friction, Drag) opposite to direction of velocity.
	\end{enumerate}
	Please note that vector arrow lengths are not perfectly proportionally accurate.
	\begin{center}
		\begin{tikzpicture}
			\pgfmathsetmacro{\MaxVelocityLength}{2}
			\pgfmathsetmacro{\MinVelocityLength}{1}
			\pgfmathsetmacro{\MaxResistanceLength}{\MaxVelocityLength*0.5}
			\pgfmathsetmacro{\MinResistanceLength}{\MinVelocityLength*0.5}
			\pgfmathsetmacro{\GravityLength}{1}
			\pgfmathsetmacro{\PathRadius}{5}
			\pgfmathsetmacro{\GliderPointRadius}{0.1}
			\coordinate(TopPos) at (0,\PathRadius);
			\coordinate(BottomPos) at (0,-\PathRadius);
			
			\draw[dashed, gray] (0,0) circle (\PathRadius);
			
			\filldraw[black] (TopPos) circle (\GliderPointRadius);
			\draw[->] (TopPos) -- ++(-\MinVelocityLength,0) node[left] {$\vec{v}$};
			\draw[->] (TopPos) -- ++(0,-\GravityLength) node[below] {$\vec{F_g}$};
			\draw[->] (TopPos) -- ++(0,-0.5) node[right] {$\vec{F_L}$};
			\draw[->] (TopPos) -- ++(\MinResistanceLength,0) node[right] {$\vec{F_R}$};
			
			\filldraw[black] (BottomPos) circle (\GliderPointRadius);
			\draw[->] (BottomPos) -- ++(\MaxVelocityLength,0) node[right] {$\vec{v}$};
			\draw[->] (BottomPos) -- ++(0,-\GravityLength) node[below] {$\vec{F_g}$};
			\draw[->] (BottomPos) -- ++(0,2) node[above] {$\vec{F_L}$};
			\draw[->] (BottomPos) -- ++(-\MaxResistanceLength,0) node[left] {$\vec{F_R}$};
		\end{tikzpicture}
	\end{center}
	\begin{figure}[H]
		\centering
		\includegraphics[width=0.8\textwidth]{Images/Dip in vertical circle.png}
		\caption{The dip section of verticle circular motion with a quadratic curve fit}
	\end{figure}
	Quadratic curve fit is applied to closely fit the lift force data in the section of the dip motion from 18.50s to 18.88s. It helps exclude random variation / noise in the data and provide consistent smoothness and symmetry. The line displayed in the center marks 18.68s, which is when the emulated glider reaches its lowest vertical height (i.e. lowest point in the dip), also aligning with the local peak in the emulated lift force at 6.29N.
	\subsection{Excellence}
	\section{Banked Corner}
	\subsection{Achieved}
	\subsection{Merit}
	\subsection{Excellence}
	\section{Additional Info}
	\subsection{Comprehensive Version History}
	Access to all prior versions of this document during process of creation is publicly available at:\\
	\url{https://github.com/NathanTaskerPersonal/AS91522}
	\subsection{Graphical Analysis Files}
	Access to all graphical analysis files are publically available at:\\
	\href{https://middletonschoolnz-my.sharepoint.com/:f:/g/personal/taskern_middleton_school_nz/EhEmw21C2L9Fn9BYUy2ccwMBn6xCUF93vtfvtT_5_rkxbA?e=Tp02lP}{middletonschoolnz-my.sharepoint.com/...}
	\subsection{\LaTeX}
	This document has been entirely written in \LaTeX; a markup language for mathematical equations and diagrams. I'm more than happy to provide plain text for the entire document for convenience in AI detection.
	\subsection{Bibliography}
\end{document}
